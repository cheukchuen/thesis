The definition of UCM metamodel and the specification of weaving algorithm described in previous chapter provide the foundation for the implementation of UCM in TouchCORE. UCM allows us to describe precise sequences graphically display a workflow model. In this chapter, we illustrate the realization of scenario models in TouchCORE through the use of UCM notation in section \ref{sec:4.1}. Then we attempt to validate our proposed approach of concern-oriented UCMs by means of case studies in section \ref{sec:4.2}. Finally, we demonstrate that concern-oriented UCMs are able to cover the workflow patterns in section \ref{sec:4.3}.

\section{UCM Implementation in TouchCORE} \label{sec:4.1}

\todo[inline]{Is there a lot of interesting details to say about your implementation in TouchCORE / EMF / Java? If yes, then keep this chapter. If not, you could add a section about the implementation to the validation chapter, as an implementation is a form of validation (in the sense that if you implement your weving algorithm, and can run it on models to compose them, and the result is "correct", i.e., the algorithm terminates and the resulting model is as expected, then your algorithm must at least not be completely flawed :)}

The project uses Java SE Development Kit 8 as the implementation language and Eclipse Modeling Framework (EMF) \cite{steinberg2008emf} as the modeling facility for building TouchCORE. To support a new language, we need to specify its metamodel through EMF.

\subsection{Model Weaving with UCMs}

\textbf{Model Extension}

\textbf{Model Reuse}

\section{Case Studies} \label{sec:4.2}

\subsection{Authentication}

\subsection{Online Payment}

\section{Workflow Patterns} \label{sec:4.3}

%\cite{van2003workflow}
%\cite{mussbacher2007evolving}
