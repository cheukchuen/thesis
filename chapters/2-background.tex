URN consists of two notations: GRL for modeling non-functional requirements, and UCM for modeling functional requirements. In this chapter, we describe UCM as part of the requirements engineering tool and its use in specifying scenarios in section \ref{sec:2.1}. Then we detail about CORE as a reuse technique with the current state of development in section \ref{sec:2.2}. Finally, we provide an overview of related work to modeling tools, applicable to the use of UCM and concern-orientation, in section \ref{sec:2.3}.

\section{Use Case Map (UCM)} \label{sec:2.1}

This section describes UCM in detail.
\todo[inline]{Brief overview and what UCMs are used for. Maybe one example UCM (that is used later on)}.
\todo[inline]{UCM metamodel (or a simplified version of it)}.
\todo[inline]{UCM Weaving? Could also be explained later before you talk about your new weaving algorithm}.

\section{Concern-oriented Reuse (CORE)} \label{sec:2.2}

This section describes CORE in detail.
\todo[inline]{You can reuse some of our text from previous papers here. Focus on what is really interesting for us in this thesis, i.e., interfaces, features + realization models, customization mappings, weaving algorithm. You should also show the CORE metamodel here (or at least a simplified version of it that is used as a basis for the integration in chapter 3. You can use some of the text from our SoSyM paper.}.

\subsection{Reusable Aspect Models (RAM)}

\todo[inline]{No need to explain too much here, since we don't use RAM really in the rest of the thesis, no? Explain mostly that RAM is the only language currently integrated with CORE, and that the models are about design. You could use class diagrams to give an example of weaving, if you think it is necessary}.

\section{Review of Related Modeling Tools} \label{sec:2.3}

Literature review.
\todo[inline]{Most of your related work will be Gunter's AoUCM stuff, which you probably mention above already. jUCMNav as well, but this you could also mention above when you talk about UCMs.}.

Last paragraph leads to implementation chapter.
\todo[inline]{I really think the next chapter should not be about implementation, but about the "theory" behind the UCM + CORE integration}.

