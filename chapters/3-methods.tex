In this chapter, we describe the integration of UCM as a new modeling language into existing modeling languages of TouchCORE. We describe the definition of UCM metamodel in section \ref{sec:3.1}. We also present the weaving algorithm for UCM in the context of CORE in section \ref{sec:3.2}.
\todo[inline]{Yes that's it. Don't talk about implementation. The integration you are describing here, i.e., metamodel and weaver, constitute the conceptual integration. Here you explain why you are doing things this way, why we are mapping connection points, for example, why we need  slightly different weavings for extend / reuse. Talk about TouchCORE / EMF later on in the following chapter. }

\section{UCM Metamodel} \label{sec:3.1}

%\begin{figure}[h]
%	\centering
%	\includegraphics[scale=0.5]{graph_a}
%	\caption{UCM metamodel: }
%	\label{fig:3.1}
%\end{figure}

\todo[inline]{Here you can describe your UCM metamodel, i.e., how and why it differs from the original shown in chapter 2, and how you subclassed the important CORE metamodel classes to integrate your UCMs with CORE.}

We follow the URN specification \cite{itu2012151} closely in designing the UCM metamodel. 

Figure \ref{fig:3.1} 

The basic structure of a UCM 

%MDE encourages the use of models as part of the development process. A model is an abstract representation of a system, and the modeling language used to specify the model can be described using a metamodel. A metamodel is yet another abstraction, describing concepts of a model. As such, a particular model always conforms to a unique metamodel. Metamodels can be written using the Meta-Object Facility (MOF), a standard proposed by Object Management Group (OMG) for MDE.

%We follow the URN specification \cite{itu2012151} closely in designing the UCM metamodel. 

\section{Weaver} \label{sec:3.2}

\todo[inline]{Again, maybe we need a short intro here to explain why weaving is needed. I assume there is also a common algorithm to extensions and reuses, or are they completely separate? Then describe the algorithm }

\todo[inline]{Explain that we need to specify composition specifications, i.e., mappings.}

\subsection{Responsibility Mapping}

\subsection{Connecting Point Mapping}
