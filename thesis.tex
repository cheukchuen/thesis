\documentclass[12pt,letterpaper]{report}
\usepackage[margin=1in,includeheadfoot]{geometry}
\usepackage[pdftex,
pdfauthor={Cheuk Chuen Siow},
pdftitle={Concern-Oriented Use Case Maps},
pdfsubject={Computer Science},
pdfkeywords={CORE, UCM},
hidelinks,bookmarksnumbered]{hyperref}
\usepackage[figure,table]{hypcap}
\usepackage[nottoc]{tocbibind}
\usepackage[titletoc]{appendix}
\usepackage[utf8]{inputenc}
\usepackage[T1]{fontenc}
\usepackage{todonotes}
\usepackage{algorithm}
\usepackage{algpseudocode}
\usepackage{subfig}
\usepackage{enumitem}

\usepackage{fancyhdr}
\pagestyle{fancy}
\fancyhf{}
\rhead{\textit{\leftmark}}
\cfoot{\thepage}
\renewcommand{\headrulewidth}{0pt}

\usepackage[backend=biber,sorting=none]{biblatex}
\addbibresource{references.bib}

\usepackage{graphicx}
\graphicspath{ {images/} }

\usepackage{setspace}
\doublespacing

\usepackage{datetime}
\newdateformat{monthyeardate}{\monthname[\THEMONTH] \THEYEAR}
\newdateformat{yeardate}{\THEYEAR}

\usepackage{sectsty}
\chapterfont{\vspace*{-17.5mm}\large\sc\centering}
\chaptertitlefont{\centering}
\subsubsectionfont{\centering}

\usepackage{chngcntr}
\counterwithout{footnote}{chapter}
\renewcommand{\thefootnote}{\fnsymbol{footnote}}

\newcommand*{\cls}{\fontfamily{cmss}\selectfont}

\newcommand{\jkReplace}[2]{\textcolor{red}{JK replaced: }\textcolor{blue}{#1}{\textcolor{red}{\ by }\textcolor{green}{#2}}}

%----------------
% Title Page
%----------------
\begin{document}
	
	\begin{titlepage}
		\begin{center}
			\vspace*{1in}
			
			\Huge
			Concern-Oriented Use Case Maps
			
			\vfill
			
			\large
			Cheuk Chuen Siow\\
			School of Computer Science\\
			McGill University, Montréal
			
			\vfill
			
			\monthyeardate\today
			
			\vfill
			
			A thesis submitted to McGill University in partial fulfillment of the requirements of the degree of Master of Science
			
			\vfill
			
			\textcopyright{} Cheuk Chuen Siow, \yeardate\today
			
			\vspace*{1in}
		\end{center}
	\end{titlepage}
	
	%----------------
	% Front Matter
	%----------------
	\clearpage
	\pagenumbering{roman}
	\setcounter{page}{2}
	
	\chapter*{Abstract}
	\markboth{\uppercase{Abstract}}{}
	\addcontentsline{toc}{chapter}{Abstract}
	
	Concern-Oriented Reuse (CORE) is a reuse paradigm that extends model-driven engineering with advanced modularization, goal modeling, and software product lines. Previous work enables modeling with CORE at the design level using Reusable Aspect Models (RAM). Requirements elicitation is also a crucial aspect of software development process, and one of the visual notation that expresses use cases as graphical workflows is Use Case Maps (UCM). UCM bridges the gap between requirements and design, and is part of the User Requirements Notation (URN) for scenario modeling. This thesis addresses the need for enabling scenario modeling in CORE. Based on Aspect-Oriented Use Case Maps (AoUCM), we introduce a novel technique that applies advanced separation of concerns for model-driven requirements elicitation with use cases---Concern-Oriented Use Case Maps (CoUCM). We define a metamodel for CoUCM that derives from the CORE metamodel, and formulate the weaving algorithm for CoUCM model composition. We then implement CoUCM in the TouchCORE tool as proof of concept. We present a working application of scenario modeling with TouchCORE, in which we further validate our implementation through case studies and workflow patterns. Validation shows that CoUCM is able to model certain requirements concerns in a reusable way, and that they can then easily be applied to different reuse contexts.
	
	\clearpage
	
	\chapter*{Abrégé}
	\markboth{\uppercase{Abrégé}}{}
	\addcontentsline{toc}{chapter}{Abrégé}
	
	\emph{Concern-Oriented Reuse} (CORE)\footnote{Traduit en Fran{\c c}ais: "La R{\'e}utilisation Orient{\'e}-Pr{\'e}occupations"} est une nouvelle approche de r{\'e}utilisation qui s'appuie sur l'ing{\'e}nierie dirig{\'e}e par les mod{\`e}les (IDM), la modularisation avanc{\'e}e, la mod{\'e}lisation des besoins, et les lignes de produits logiciels. Jusqu'{\`a} pr{\'e}sent, CORE permettait seulement la mod{\'e}lisation de conceptions avec le langage de mod{\'e}lisation \emph{Reusable Aspect Models} (RAM). Cependant, la mod{\'e}lisation des besoins est une partie fondamentale de l'approche IDM. \emph{Use Case Maps} (UCM) est un langage de mod{\'e}lisation qui permet d'exprimer les cas d'utilisations visuellement en tant que flux de travails. UCM permets de formaliser les cas d'utilisations textuels et donc informels, et ainsi passer plus facilement des mod{\`e}les de besoins au mod{\`e}les de conc{\'e}ption. Ce travail de th{\`e}se int{\`e}gre UCM avec l'approche de r{\'e}utilisation CORE. En s'inspirant de la t{\'e}chnique Aspect-Oriented Use Case Maps (AoUCM), nous proposons une extension de UCM orient{\'e}-pr{\'e}occupations nomm{\'e}e \emph{Concern-Oriented Use Case Maps} (CoUCM). Nous d{\'e}finissons un meta mod{\`e}le pour CoUCM qui s'int{\`e}gre avec le m{\'e}ta mod{\`e}le de CORE, et nous proposons des algorithmes de tissage qui permettent la composition de mod{\`e}les UCM venant de diff{\'e}rents pr{\'e}occupations ou charact{\'e}ristiques. Une impl{\'e}mentation prototype de CoUCM dans l'outil de mod{\'e}lisation TouchCORE est {\'e}galement d{\'e}crite. Pour valider l'expressivit{\'e} de notre solution, nous montrons comment CoUCM s'applique {\`a} plusieurs cas d'utilisation, qui d{\'e}montrent que CoUCM permet de modulariser et rendre r{\'e}utilisable certaines pr{\'e}occupations, tel que par exemple les diff{\'e}rents int{\'e}ractions de payment.	
	
	\clearpage
	
	\chapter*{Acknowledgements}
	\markboth{\uppercase{Acknowledgements}}{}
	\addcontentsline{toc}{chapter}{Acknowledgements}
	
	Foremost, I would like to express my sincere gratitude to Professor Jörg Kienzle and Professor Gunter Mussbacher, my research supervisors, for their patient guidance and insightful comments of this research work. This thesis would not be possible without their supervision.
	
	I thank my fellow colleagues from the Software Engineering Lab for their encouragement and stimulating discussions that challenged me intellectually.
	
	Last but not least, I would like to thank my family and friends around the world for their support throughout my study.
	
	% Content
	\renewcommand{\contentsname}{Table of Contents}
	\tableofcontents
	\listoffigures
	%\listoftables
	\listofalgorithms
	\addcontentsline{toc}{chapter}{List of Algorithms}
	
	\clearpage
	
	\chapter*{List of Acronyms}
	\markboth{\uppercase{List of Acronyms}}{}
	\addcontentsline{toc}{chapter}{List of Acronyms}
	
	\begin{description}[noitemsep,style=multiline,leftmargin=3cm]
		\item[ADORE] Activity moDel supOrting oRchestration Evolution
		\item[AoUCM] Aspect-Oriented Use Case Maps
		\item[AoURN] Aspect-Oriented User Requirements Notation
		\item[BPEL] Business Process Execution Language
		\item[BPMN] Business Process Model and Notation
		\item[CORE] Concern-Oriented Reuse
		\item[CoUCM] Concern-Oriented Use Case Maps
		\item[EMF] Eclipse Modeling Framework
		\item[GRL] Goal-oriented Requirements Language
		\item[GUI] Graphical User Interface
		\item[ITU] International Telecommunication Union
		\item[MDE] Model-Driven Engineering
		\item[MVC] Model-View-Controller
		\item[RAM] Reusable Aspect Models
		\item[SOA] Service-Oriented Architecture
		\item[SoC] Separation of Concerns
		\item[SPL] Software Product Lines
		\item[UCM] Use Case Maps
		\item[URN] User Requirements Notation
		\item[XML] Extensible Markup Language
	\end{description}
	
	\clearpage
	\pagenumbering{arabic}
	
	%----------------
	% Main Body
	%----------------
	\chapter{Introduction}
	First paragraph starts off with a short history of software engineering.

Second paragraph introduces the topic of MDE.

Third paragraph discusses the current state and potential drawbacks of MDE.

Forth paragraph introduces the notion of CORE.
\todo[inline]{and how it uses aspect-oriented modelling techniques to allow complex models to be built by incrementally composing smaller, simpler models. Mention that for now only design models, i.e., models of one development phase, are supported.}

Fifth paragraph continues with TouchCORE and its current state.
\todo[inline]{maybe not needed}

Build on the motivation of having UCM as an additional model for TouchCORE.
\todo[inline]{i.e., UCMs are scenario models, typically built earlier than design models. Aspect-oriented composition has already been investigated for UCMs (reference Gunter's Ph.D. and papers). The goal of this thesis is to concernify UCMs, i.e., determine how UCMs can be integrated with the concepts of CORE, and how the weaving needs to be adapted to fit with CORE. The ultimate goal being to investigate whether real MDE, i.e., software development with models at multiple levels of abstraction and model transformations that connect them, is compatible with CORE.}

Last paragraph leads the reader to the remaining chapters.

	
	\chapter{Background} \label{ch:2}
	First paragraph opens with URN and its components.

\todo[inline]{I would start this chapter by a short paragraph that presents the outline of the chapter, i.e., that you will talk about use case maps, then CORE, then tools. Since this is the background chapter, I would also recommend to not put the motivation here... either move it to the introduction, or else at the beginning of chapter 3.}

\section{Use Case Map (UCM)}

This section describes UCM in detail.

\section{Concern-oriented Reuse (CORE)}

This section describes CORE in detail.

\subsection{Reusable Aspect Models (RAM)}

\section{Overview of Relevant Modeling Tools}

Literature review.

\section{Motivation}

Build on the motivation of having UCM as an additional model for TouchCORE.

\subsection{UCM in the Context of CORE}

Weaving, extending, and reusing UCM concern models.

Last paragraph leads to implementation chapter.

	
	\chapter{Adding Support for UCM to CORE} \label{ch:3}
	% !TEX root = ../thesis.tex

\todo[inline]{Here we need to add one paragraph that reminds the reader that while in theory CORE supports multiple modelling languages, so far only one notation has been integrated with CORE and is supported in the TouchCORE tool: the RAM modelling notation used for design modelling with class and sequence diagrams. The main goal of the thesis is to investigate whether CORE can also support a requirements specification language (in our case UCM).}
This chapter presents the corification of a modeling language for the requirements phase using the CORE metamodel. We describe the steps taken to corify UCM in Section~\ref{sec:3.1} \todo[inline]{mention here that in particular we need to determine the customization and usage interfaces for UCMs}. We also present the weaving algorithm specific for UCM in the context of CORE in Section~\ref{sec:3.2} \todo[inline]{not just any weaving algorithm, a weaving algorithm that fulfills the needs of a requirements engineer and allows him to build modular UCMs using CORE model extensions and CORE model reuses}.

\section{Corification of UCM} \label{sec:3.1}

\todo[inline]{All the necessary details about what you have accomplished are described in this subsection, but they are not presented in a "requirements-driven" way. What I mean is that I suggest you first tell the reader why something needs to be done, and then how you did it. For example, you first explain that integrating a modelling language with CORE means to allow concerns to contain models of that language. Hence we need to make it possible for UCMs to be part of concerns, i.e., serve a realization models for features. Then you can explain the details of how you achieved this in the metamodel.} Abstract and concrete classes of the CORE metamodel are utilized differently when corifying a modeling language. The abstract classes \textit{\cls COREModel}, \textit{\cls COREModelElement}, and \textit{\cls COREPattern} serve as extension points and are intended to be subclassed by a modeling language. This enables the addition of arbitrary modeling languages to CORE and also uniform treatment of pattern-based composition. The remaining abstract classes \textit{\cls COREModelComposition}, \textit{\cls COREModelElementComposition}, and \textit{\cls CORELink} are used within the CORE metamodel and seldom subclassed by a modeling language. On the contrary, concrete classes are designed to be used exactly as it is in the corified modeling languages. They provide the necessary mechanisms for model extensions and reuses, feature and impact modeling, as well as a way to implements and visualizes these concepts in its modeling tool.

\begin{figure}
	\centering
	\includegraphics[scale=0.5]{fig_3_1.pdf}
	\caption{Extension of the CORE metamodel by UCM}
	\label{fig:3.1}
\end{figure}

We follow the URN specification~\cite{itu2012151} closely in corifying the UCM metamodel. Figure~\ref{fig:3.1} shows a partial view of the corified UCM metamodel, focusing on the elements that extend the CORE metamodel through subclassing (from an existing metaclass in the modeling language to an abstract CORE metaclass \footnote{The gray elements in the figures are the classes that derived from the CORE metamodel.}). By subclassing the necessary abstract classes of the CORE metamodel, UCM is able to provide all the properties of CORE:

\begin{itemize}
	\setlength{\parskip}{0pt} \setlength{\itemsep}{0pt}
	\item A UCM model may now belong to a concern by realizing at least one of its features.
	\item A UCM realization model may now have impacts on high-level goals.
	\item A UCM model may extend another UCM model that belongs to a different feature.
	\item A UCM model may reuse another UCM model that belongs to a different concern.
\end{itemize}

\todo[inline]{What follows is all about figuring out what the best customization interface for UCMs would be, and what the things are that will be mapped. I would introduce this maybe by presenting an example base scenario, and how one might want to extend it. Then show how one would model the example base scenario with a UCM, and then show how the extended UCM scenario would look like. Then explain how you envision to allow the modeller to not have to specify the complete extended scenario, but simply "augment" the base scenario with a model extension. Then proceed by explaining the details in the metamodel. If model reuse is very different from model extension then this should also be explained separately.}
Reusing a concern from a UCM model prompts the feature selection process, by asking the feature that the UCM model realizes to reuse the other concern with the selection of feature(s) it wants to reuse. The reusing UCM model then establishes the mappings to the reused UCM model that realizes the reused features. This is achieved as follows. The root element {\cls UseCaseMap} subclasses \textit{\cls COREModel}, which makes it part of a {\cls COREConcern} (see Figure~\ref{fig:2.1}, association between {\cls COREConcern} and {\cls COREModel}). This allows a UCM to realize a feature (see Figure~\ref{fig:2.2}, {\cls COREModel} realizes {\cls COREFeature}) within a concern. Therefore, the concern can create a {\cls COREReuse} to reuse another concern. The reusing UCM then creates a {\cls COREModelReuse} that has a direct association to the created {\cls COREReuse} and a \textit{\cls COREConfiguration} that selects the desired features from the reused concern. The CORE modeling tool then composes the UCM models of the reused concern that realize the selected features to generate a single woven user-tailored UCM model of the reused concern. Mappings to the model elements of this generated model are established using the class {\cls COREMapping}, consequently allowing the reusing UCM to customize the generated UCM model of the reused concern.

\begin{figure}
	\centering
	\includegraphics[scale=0.5]{fig_3_2.pdf}
	\caption{Path nodes for corified UCM}
	\label{fig:3.2}
\end{figure}

A standard UCM consists of {\cls PathNode}, {\cls Responsibility}, and {\cls NodeConnection}. {\cls LayoutMap} is added as part of the composition to allow positioning of the elements for viewing. We omit the inclusion of certain elements such as {\cls Component}, {\cls Timer}, and {\cls FailurePoint} to limit the scope of this thesis. On the contrary, {\cls PluginBinding} is excluded on purpose since we utilize {\cls COREMapping} as our approach to bind separate UCMs to {\cls Stub}. We incorporate several changes to the path nodes to support aspect-oriented modeling and reuse. Figure~\ref{fig:3.2} illustrates the addition of {\cls Anything} and {\cls ConnectingPoint}, as well as a directed association from {\cls Stub} to {\cls COREModelReuse}, to the UCM metamodel.

\textbf{\cls Anything:} We included the {\cls Anything} pointcut element from the extended AoUCM metamodel~\cite{mussbacher2011aspect}. {\cls Anything} acts as a wild card and can represent a subset of nodes in a path. This is useful for facilitating complex model weaving, as it allows any sequence of UCM model elements, including an empty sequence, to be matched.
\todo[inline]{Is this similar to the *-box in RAM that refers to the "original" behaviour of an advised sequence diagram? It might be interesting here to point out the similarities.}

\textbf{\cls ConnectingPoint:} We established a new path element to the metamodel. {\cls ConnectingPoint} is used to replace {\cls PluginBinding} and serves as an intermediate node that represents either a {\cls StartPoint} or an {\cls EndPoint}. By default, an actual start or end point within a UCM does not have a reference to a stub, hence the default value for {\cls StubReference} is {\cls None}. Instead, when we have a {\cls NodeConnection} that connects an element with a stub, then a hidden connecting point is automatically attached to the node connection (and deleted upon removal of the connection). Each node connection can have at most two connecting points if both the source and target nodes of the connection are stubs. Incoming connection to a stub generates a hidden end point with the value of {\cls stubRef} set to {\cls Target}, whereas outgoing connection from a stub generates a hidden start point with the value of {\cls stubRef} set to {\cls Source}. These hidden points allow us to define composition specifications through customization mappings.

\begin{figure}
	\centering
	\includegraphics[scale=0.5]{fig_3_3.pdf}
	\caption{Customization mappings for corified UCM}
	\label{fig:3.3}
\end{figure}

Since we are using {\cls COREMapping} to specify customizations, it is necessary for {\cls UCMModelElement} to subclass \emph{\cls COREModelElement}. That way, all subclasses of {\cls UCMModelElement} (i.e., {\cls PathNode} and {\cls Responsibility}) can be used as source and destination classes for {\cls COREMapping}. As shown in Figure~\ref{fig:3.3}, we defined the composition specifications for specific UCM model elements: {\cls Responsibility} and {\cls ConnectingPoint}. They were selected so that we can compose UCM models based on the mappings of these elements. This leads us to the next section where we describe in detail how model composition is achieved through weaving.

\todo[inline]{I read up to here, but will continue soon}

\section{UCM Weaving} \label{sec:3.2}

As explained in Section~\ref{sec:2.1}, the role of the weaver is to facilitate model extensions and reuses. We offer two options when mapping elements between UCMs: (i) direct mapping of responsibilities; and (ii) cross mapping of connecting points. Cross mapping is necessary because of the nature of start and end points, where a start point of a UCM maps to an end point of a stub, and vice versa. A stub can be perceived as being superimposed with an end point followed by a start point, and those points collapsed into a point that is the stub \cite{buhr1995use}. Here, the hidden end point of a stub represents the incoming connection and it signifies the end of the sequence before the stub, and the hidden start point of a stub represents the outgoing connection and it signifies the start of the sequence after the stub. Both options have different procedures when weaving.

\subsection{Weaving Algorithm}

The algorithms presented here are specific for single weaving, meaning that a composition is performed from one model ($UCM_1$) to another model ($UCM_2$). This action can be chained together with other compositions, even with the hierarchical structure of the concern features. Here, we specify the subscript $_1$ for the model elements of a UCM the weaver composes from, and the subscript $_2$ for the model elements of a UCM the weaver composes to. $UCM_1$ and $UCM_2$ are merged prior to weaving, retaining all the path nodes and node connections from both models. Then the weaver iterates through the available composition specifications and executes the algorithms based on the specific type of mapping. The output of the woven model results in the amalgamation of UCMs based on the composition specification defined by the designer of the models, as well as the selected features of the concern by the user.

\subsubsection{Responsibility Mapping} \label{sec:3.2.1.1}

\begin{algorithm}
    \caption{Weaving Algorithm: Responsibility Mapping}
    \label{alg:1}
	\begin{algorithmic}[1]
	    \Function{WeaveResponsibilityMapping}{$ucm, composition$}
			\State $node_1\gets$ get first node of composition mapping ($from$)
			\State $node_2\gets$ get second node of composition mapping ($to$)
			\State mark $node_2$ as visited
			\State indicate start point has not been encountered
			\State indicate end point has not been encountered
			\State call \Call{TraverseToSource}{$ucm, node_2, node_1$}
			\State call \Call{TraverseToTarget}{$ucm, node_2, node_1$}
			\State remove $node_1$ from $ucm$
		\EndFunction
		
		\Function{TraverseToSource}{$ucm, node_2, node_1$}
			\For{each predecessor of $node_2$}
				\State $sourceNode\gets$ get source node of predecessor
				\If{linkage exists from previous mapping} \label{alg:1.1}
					\State set the source of $node_2$'s connection to $node_1$'s predecessor
					\State disable linkage
					\State skip this loop
				\EndIf \label{alg:1.2}
				\If{$sourceNode$ is visited} \label{alg:1.3}
					\State skip this loop
				\ElsIf{$sourceNode$ is not {\cls Anything}}
					\State mark $sourceNode$ as visited
				\EndIf \label{alg:1.4}
				\If{$sourceNode$ is {\cls StartPoint} and start point is not encountered} \label{alg:1.5}
					\If{visibility of $sourceNode$ is {\cls Concern}}
						\State set the source of $node_2$'s connection to $node_1$'s predecessor
						\State remove $sourceNode$ from $ucm$
						\State indicate start point has been encountered
					\EndIf \label{alg:1.6}
				\ElsIf{$sourceNode$ is {\cls Anything}} \label{alg:1.7}
					\State set the source of $node_2$'s connection to $node_1$'s predecessor
					\If{$sourceNode$ does not have any predecessor}
						\State remove $sourceNode$ from $ucm$
					\EndIf \label{alg:1.8}
				\Else
					\State recursively call \Call{TraverseToSource}{$ucm, sourceNode, node_1$} \label{alg:1.9}
				\EndIf
			\EndFor
		\EndFunction
		
		\algstore{alg1}
	\end{algorithmic}
\end{algorithm}

\begin{algorithm}                     
	\begin{algorithmic}[1]
		\algrestore{alg1}
		
		\Function{TraverseToTarget}{$ucm$, $node_2$, $node_1$}
			\For{each successor of $node_2$}
				\State $targetNode\gets$ get target node of successor
				\If{$targetNode$ is visited} \label{alg:1.10}
					\State skip this loop
				\ElsIf{$targetNode$ is not {\cls Anything}}
					\State mark $targetNode$ as visited
				\EndIf \label{alg:1.11}
				\If{$targetNode$ is {\cls Endpoint} and end point is not encountered} \label{alg:1.12}
					\If{visibility of $targetNode$ is {\cls Concern}}
						\State set the target of $node_2$'s connection to $node_1$'s successor
						\State remove $targetNode$ from $ucm$
						\State indicate end point has been encountered
					\EndIf \label{alg:1.13}
				\ElsIf{$targetNode$ is {\cls Anything}} \label{alg:1.14}
					\State set the target of $node_2$'s connection to $node_1$'s successor
					\If{$targetNode$ does not have any successor}
						\State remove $targetNode$ from $ucm$
					\EndIf \label{alg:1.15}
				\ElsIf{$targetNode$ exists in compositions mapping ($to$)} \label{alg:1.16}
					\State copy node connection of successor
					\State set the target of copied connection to $node_1$'s successor
					\State add the copied connection to $ucm$
					\State enable linkage to next mapping \label{alg:1.17}
				\Else
					\State recursively call \Call{TraverseToTarget}{$ucm, targetNode, node_1$} \label{alg:1.18}
				\EndIf
			\EndFor
		\EndFunction
	\end{algorithmic}
\end{algorithm}

Mapping with responsibilities allows for model extensions between parent and child UCMs. Composition specification can be defined by mapping from a parent UCM's responsibility to a child UCM's responsibility. Algorithm~\ref{alg:1} illustrates the procedure of weaving for responsibility mappings. The function \emph{WeaveResponsibilityMapping} initiates the process by identifying the mapped responsibilities (\emph{from} $UCM_1$ \emph{to} $UCM_2$), and traversal begins from the point of $responsibility_2$ in both directions: (i) toward predecessors until start point encountered; and (ii) toward successors until end point encountered. A UCM is represented as a directed graph, with possible cycles via {\cls OrFork}s and {\cls OrJoin}s. As such, we implemented a depth-first search approach for traversing the graph through recursion (lines~\ref{alg:1.9} and~\ref{alg:1.18}), and a mechanism to determine whether a node has been explored (lines~\ref{alg:1.3}-\ref{alg:1.4} and~\ref{alg:1.10}-\ref{alg:1.11}).

Furthermore, we allow multiple consecutive mappings between two UCM models. The path of a UCM may consist of mapped responsibilities interspersed with other path nodes. While traversing forward, lines~\ref{alg:1.16}-\ref{alg:1.17} handle the next mapped responsibility. If exist, forward traversal stops for this specific composition and appropriate nodes are connected between $UCM_1$ and $UCM_2$. For subsequent mappings, lines~\ref{alg:1.1}-\ref{alg:1.2} handle the linkage from previous mappings, and backward traversal stops at the point of mapped responsibilities and appropriate nodes are connected between $UCM_1$ and $UCM_2$. This pattern continues until the weaver reaches end point, whereby the predecessor of $UCM_2$'s end point connects to the successor of mapped responsibility ($from$) $UCM_1$ and this end point gets deleted (lines~\ref{alg:1.12}-\ref{alg:1.13}). Same goes for backward traversal until the weaver reaches start point (lines~\ref{alg:1.5}-\ref{alg:1.6}). Lastly, mapped responsibility ($to$) $UCM_2$ retains while mapped responsibility ($from$) $UCM_1$ gets deleted for the final woven UCM model.

UCM may have multiple start points merging to a path, or a path may branch to multiple end points. In this case, we allow a start or end point to set its visibility level. By default, connecting point is given the visibility of {\cls Concern} that signifies the start or end point is only visible when viewing a UCM model for a specific feature of a concern, but disappears after the composition process. The other option is {\cls Public} for global visibility and is used to retain the start or end point even after the composition process---the weaver would just ignore {\cls Public} connecting points and proceed to other branches. This feature is useful in defining multiple entry points, or alternative exit strategies, for a scenario.

Complex scenario model composition is also possible with the help of {\cls Anything}. An anything node can represent a subset of nodes in a path and is commonly used in $UCM_2$ to capture the actual nodes that are specified in $UCM_1$. If an anything node is encountered during traversal, lines~\ref{alg:1.7}-\ref{alg:1.8} and~\ref{alg:1.14}-\ref{alg:1.15} signals the end of exploration and treat it as an end point. The difference is that the algorithm checks whether the anything node is still connected to other nodes before removal. This is necessary because an anything node has a predecessor node and a successor node, and typically surrounded by forks and joins (loop cycle). Both sides have to be traversed and dealt with before removing the anything node from the woven model.

\subsubsection{Connecting Point Mapping}

\begin{algorithm}
	\caption{Weaving Algorithm: Connecting Point Mapping}
	\label{alg:2}
	\begin{algorithmic}[1]
		\Function{WeaveConnectingPointMapping}{$ucm$, $composition$}
			\State $node_1\gets$ get first node of composition mapping ($from$)
			\State $node_2\gets$ get second node of composition mapping ($to$)
			\If{$node_1$ is {\cls StartPoint}}
				\If{$node_1$ is connected to a stub} \label{alg:2.1}
					\State call \Call{ExtendingStub\_End}{$ucm, node_2, node_1$} \label{alg:2.2}
				\ElsIf{$node_2$ is connected to a stub} \label{alg:2.3}
					\State call \Call{ReusingStub\_Start}{$ucm, node_2, node_1$} \label{alg:2.4}
				\EndIf
			\ElsIf{$node_1$ is {\cls EndPoint}}
				\If{$node_1$ is connected to a stub} \label{alg:2.5}
					\State call \Call{ExtendingStub\_Start}{$ucm, node_2, node_1$} \label{alg:2.6}
				\ElsIf{$node_2$ is connected to a stub} \label{alg:2.7}
					\State call \Call{ReusingStub\_End}{$ucm, node_2, node_1$} \label{alg:2.8}
				\EndIf
			\EndIf
		\EndFunction
		
		\Function{ExtendingStub\_End}{$ucm, node_2, node_1$} \label{alg:2.9}
			\State $source_2\gets$ get source node of $node_2$
			\State $source_1\gets$ get source node of $node_1$ via stub connection
			\State $target_1\gets$ get target node of $node_1$ via stub connection
			\State call \Call{MergePaths}{$source_2, source_1, target_1, node_1, node_1$}
			\State remove $node_2$ from $ucm$
		\EndFunction \label{alg:2.10}
		
		\Function{ReusingStub\_Start}{$ucm, node_2, node_1$} \label{alg:2.11}
			\State $target_1\gets$ get target node of $node_1$
			\State $target_2\gets$ get target node of $node_2$ via stub connection
			\State $source_2\gets$ get source node of $node_2$ via stub connection
			\State call \Call{SplitPaths}{$target_1, target_2, source_2, node_2, node_1$}
			\State remove $node_1$ from $ucm$
		\EndFunction \label{alg:2.12}
		
		\Function{ExtendingStub\_Start}{$ucm, node_2, node_1$} \label{alg:2.13}
			\State $target_2\gets$ get target node of $node_2$
			\State $target_1\gets$ get target node of $node_1$ via stub connection
			\State $source_1\gets$ get source node of $node_1$ via stub connection
			\State call \Call{SplitPaths}{$target_2, target_1, source_1, node_1, node_1$}
			\State remove $node_2$ from $ucm$
		\EndFunction \label{alg:2.14}
		
		\algstore{alg2}
	\end{algorithmic}
\end{algorithm}

\begin{algorithm}                     
	\begin{algorithmic}[1]
		\algrestore{alg2}
		
		\Function{ReusingStub\_End}{$ucm, node_2, node_1$} \label{alg:2.15}
			\State $source_1\gets$ get source node of $node_1$
			\State $source_2\gets$ get source node of $node_2$ via stub connection
			\State $target_2\gets$ get target node of $node_2$ via stub connection
			\State call \Call{MergePaths}{$source_1, source_2, target_2, node_2, node_1$}
			\State remove $node_1$ from $ucm$
		\EndFunction \label{alg:2.16}
		
		\Function{SplitPaths}{$target, target', source', node, node'$}
			\If{$target'$ is {\cls Stub}} \label{alg:2.17}
				\State mark $target'$ as removable stub
				\State set target node of $node$'s stub connection to $target$ \label{alg:2.18}
			\ElsIf{$target'$ is {\cls AndFork} or {\cls OrFork}} \label{alg:2.19}
				\State create node connection between $target'$ and $target$ \label{alg:2.20}
			\Else \label{alg:2.21}
				\State $referenceStub\gets$ get target node of $node'$ via stub connection
				\State $forkNode\gets$ \textbf{if} $referenceStub$ is dynamic \textbf{then} create {\cls AndFork} \textbf{else} {\cls OrFork}
				\State place $forkNode$ in between $source'$ and $target'$
				\State create node connection between $forkNode$ and $target'$
				\State create node connection between $forkNode$ and $target$
				\State set target node of $node$'s stub connection to $forkNode$
			\EndIf \label{alg:2.22}
		\EndFunction
		
		\Function{MergePaths}{$source, source', target', node, node'$}
			\If{$source'$ is {\cls Stub}} \label{alg:2.23}
				\State mark $source'$ as removable stub
				\State set source node of $node$'s stub connection to $source$ \label{alg:2.24}
			\ElsIf{$source'$ is {\cls AndJoin} or {\cls OrJoin}} \label{alg:2.25}
				\State create node connection between $source$ and $source'$ \label{alg:2.26}
			\Else \label{alg:2.27}
				\State $referenceStub\gets$ get source node of $node'$ via stub connection
				\State $joinNode\gets$ \textbf{if} $referenceStub$ is synchronizing \textbf{then} create {\cls AndJoin} \textbf{else} {\cls OrJoin}
				\State place $joinNode$ in between $source'$ and $target'$
				\State create node connection between $source'$ and $joinNode$
				\State create node connection between $source$ and $joinNode$
				\State set source node of $node$'s stub connection to $joinNode$
			\EndIf \label{alg:2.28}
		\EndFunction
	\end{algorithmic}
\end{algorithm}

Mapping with connecting points allows for model extensions between parent and child UCMs and also model reuses from UCMs of other concerns. Algorithm~\ref{alg:2} illustrates the procedure of weaving for connecting point mappings. The function \emph{WeaveConnectingPointMapping} initiates the process by identifying the mapped connecting points (\emph{from} $UCM_1$ \emph{to} $UCM_2$), and determine the type of composition to be performed based on whether the connecting points mapped from $UCM_1$ are attached to a stub or not. If the mapped start and end points from $UCM_1$ are attached to a stub (lines~\ref{alg:2.1}-\ref{alg:2.2} and~\ref{alg:2.5}-\ref{alg:2.6}), it means that the connecting points are hidden and belong to a stub in $UCM_1$ and are mapped to actual end and start points of $UCM_2$, respectively (cross mapping). This type of composition is model extension. Vice versa for model reuse (lines~\ref{alg:2.3}-\ref{alg:2.4} and~\ref{alg:2.7}-\ref{alg:2.8}).

Model extension for stubs work differently compared with responsibilities. No traversal is required since there is no need to explore the whole graph, but the composition specification requires exactly two connecting point mappings for each stub to be complete---one for the start point and the second for end point. The weaver first obtain the pair of mappings for the stub. The initial mapping usually maps the end point of a stub \footnote{The end point of a stub symbolizes incoming node connection to the stub.} to the start point of a UCM, and the weaver executes lines~\ref{alg:2.9}-\ref{alg:2.10}. The second mapping usually maps the start point of a stub \footnote{The start point of a stub symbolizes outgoing node connection from the stub.} to the end point of a UCM, and the weaver executes lines~\ref{alg:2.13}-\ref{alg:2.14}.

Model reuse, on the other hand, operates in reversed orientation---obtaining the pairs of mappings that mapped the start and end points of $UCM_1$ to the connecting points of a stub that is automatically generated in $UCM_2$ when reusing $UCM_1$. To be precise, the automatically generated stub is always a static stub so that it can only hold a single UCM that originates from the reused concern. The weaver then executes lines~\ref{alg:2.11}-\ref{alg:2.12} and~\ref{alg:2.15}-\ref{alg:2.16}, respectively.

The execution procedure for both extension and reuse involves replacing a stub with plug-ins (sub-UCMs). Depending on the type of stub, it can bind either a single plug-in or multiple plug-ins. When facing a single plug-in bound to a stub, the weaver simply connects the nodes adjacent to the stub and nodes adjacent to the connecting points of a UCM, followed by the removal of the connecting points and the stub from the woven model (lines~\ref{alg:2.17}-\ref{alg:2.18} and~\ref{alg:2.23}-\ref{alg:2.24}). If there are two plug-ins bound to a stub, the weaver creates branches to link the two UCMs as parallel paths via fork and join nodes (lines~\ref{alg:2.21}-\ref{alg:2.22} and~\ref{alg:2.27}-\ref{alg:2.28}). The type of forks and joins being created is dependent on the type of stub. Synchronizing/blocking stubs produce branches that consist of {\cls AndFork} and {\cls AndJoin}, dynamic stubs produce branches that consist of {\cls AndFork} and {\cls OrJoin}, and static stubs produce branches that consist of {\cls OrFork} and {\cls OrJoin}. This process is also known as semantic flattening \cite{itu2012151}. Additional plug-ins bound to a stub are linked via the created forks and joins (lines~\ref{alg:2.19}-\ref{alg:2.20} and~\ref{alg:2.25}-\ref{alg:2.26}).

	
	\chapter{Validation} \label{ch:4}
	The definition of UCM metamodel and the specification of weaving algorithm described in previous chapter provide the foundation for the implementation of UCM in TouchCORE. UCM allows us to describe precise sequences graphically display a workflow model. In this chapter, we illustrate the realization of scenario models in TouchCORE through the use of UCM notation in section \ref{sec:4.1}. Then we attempt to validate our proposed approach of concern-oriented UCMs by means of case studies in section \ref{sec:4.2}. Finally, we demonstrate that concern-oriented UCMs are able to cover the workflow patterns in section \ref{sec:4.3}.

\section{UCM Implementation in TouchCORE} \label{sec:4.1}

\todo[inline]{Is there a lot of interesting details to say about your implementation in TouchCORE / EMF / Java? If yes, then keep this chapter. If not, you could add a section about the implementation to the validation chapter, as an implementation is a form of validation (in the sense that if you implement your weving algorithm, and can run it on models to compose them, and the result is "correct", i.e., the algorithm terminates and the resulting model is as expected, then your algorithm must at least not be completely flawed :)}

The project uses Java SE Development Kit 8 as the implementation language and Eclipse Modeling Framework (EMF) \cite{steinberg2008emf} as the modeling facility for building TouchCORE. To support a new language, we need to specify its metamodel through EMF.

\subsection{Model Weaving with UCMs}

\textbf{Model Extension}

\textbf{Model Reuse}

\section{Case Studies} \label{sec:4.2}

\subsection{Authentication}

\subsection{Online Payment}

\section{Workflow Patterns} \label{sec:4.3}

%\cite{van2003workflow}
%\cite{mussbacher2007evolving}

	
	\chapter{Conclusion} \label{ch:5}
	Requirements elicitation forms a fundamental piece of software engineering and is typically performed at the initial phase of the software development process. This thesis introduces CoUCM that consolidates scenario modeling with advanced SoC, MDE, and SPL. Along with the already existing GRL support for goal modeling in CORE, the addition of UCM offers a complete URN package for requirements engineering in CORE.

The proof of concept implementation of CoUCM in TouchCORE demonstrates the project's feasibility. TouchCORE is a modeling tool that provides an intuitive interface for concern-oriented software design. Both CoUCM and RAM allow TouchCORE users to model at different level of abstractions, covering the requirements and design phases. In addition, modelers can now populate the TouchCORE library with reusable scenario models encoding essential recurring requirements concerns (e.g., functional units, workflow patterns, etc).

This chapter summarizes the work contributed to the thesis in Section~\ref{sec:5.1}. Limitations of the work exists, however, and serve as potential future work for improvements, which we discuss in Section~\ref{sec:5.2}.

\section{Summary of Work} \label{sec:5.1}

\begin{itemize}
	
	\item We defined the metamodel for CoUCM in Section~\ref{sec:3.1}. CORE now supports UCM as an additional modeling language. The metamodel is open to expansion and this allows additional UCM features to be added in the future.
	
	\item We introduced the weaving algorithm for composing CoUCMs in Section~\ref{sec:3.2}. The UCM weaver is capable of extending UCM model of a feature with another UCM model of a subfeature within a concern, allowing requirements engineers to modularize scenarios according to concern features. The weaver is also capable of reusing existing scenario concerns of interest from another concern, allowing requirements engineers to build larger model applications from smaller concerns. These two capabilities adhere to the principles of advanced SoC that is fundamental to the practice of model-driven software development.
	
	\item We implemented the CoUCM metamodel and weaving algorithm in the TouchCORE tool in Section~\ref{sec:4.1}. TouchCORE users now have the choice to realize scenario models, in addition to design models, for any features of a concern. TouchCORE provides the necessary functionalities for the users to build UCMs with touch gestures. Models are displayed via the tool's graphical interface to illustrate the causal relationships among the activities intuitively, and woven UCM models merged by the model composition process can be easily understood when visualized.
	
	\item We demonstrated the expressiveness and reuse potential of CoUCMs through case studies, with two concerns in Section~\ref{sec:4.2} and two workflow patterns in Section~\ref{sec:4.3}. The results validated the proposed CoUCM method, allowing requirements engineers to utilize the CORE concepts as a framework to build large-scale scenario models.
	
\end{itemize}

\section{Future Work} \label{sec:5.2}

The current UCM metamodel that we support in CORE does not cover all the model elements defined in the UCM standard. In particular, the use of {\cls ResponsibilityRef}s should refer to a particular {\cls Responsibility} definition from multiple reference points that belong to other UCMs, as well as the use of {\cls Component}s to model the architectural structure of a system. Implementation of {\cls Component} to CoUCM poses some difficulties especially when taking model composition into account as responsibilities within a component are bound to the component; we may also have to support mappings of components and this adds complexity to the weaving algorithm. Several other model elements including waiting place, timer, failure point, and abort could be added to the CoUCM metamodel to complete the standard UCM features.

Similarly, the implementation of scenario modeling to TouchCORE is in the alpha phase. TouchCORE features such as traceability and model validation could be implemented to allow for a better scenario modeling experience. Path drawing could be improved as current implementation uses straight lines to connect path nodes; splines would work well if supported by TouchCORE GUI. One of the jUCMNav tool's features is the path traversal mechanism~\cite{kealey2007enhanced2}. If implemented in TouchCORE, this mechanism allows for UCM analysis and is particularly useful in evaluating scenario variables when traversing paths.

Supplementary work to the CORE base design is needed to allow a more seamless integration of multiple modeling languages. This leads to the question that begs to be investigated---whether actual MDE, i.e., software development with models at multiple levels of abstraction and model transformations that connect them, is compatible with CORE. Since this is one of the early works (after RAM) that extends a modeling language (UCM) with concern-orientation, future addition of modeling languages to CORE can refer to this work as reference. We hope that this work would motivate future studies to further improve the CORE paradigm.

	
	%----------------
	% End Matter
	%----------------
	\renewcommand\bibname{References}
	\printbibliography[heading=bibintoc]
	
	\begin{appendices}
		
		\chapter{Complete Metamodels} \label{ch:A}
		\section{CORE Metamodel}

\begin{figure}[h]
	\centering
	\includegraphics[scale=0.55]{fig_a_1.pdf}
	\caption{Abstract grammar: CORE metamodel overview}
	\label{fig:a.1}
\end{figure}

\section{CoUCM Metamodel}

\begin{figure}[h]
	\centering
	\includegraphics[scale=0.55]{fig_a_2.pdf}
	\caption{Abstract grammar: CoUCM metamodel overview}
	\label{fig:a.2}
\end{figure}

		
	\end{appendices}
	
\end{document}
