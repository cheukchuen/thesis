First paragraph starts off with a short history of software engineering.

Second paragraph introduces the topic of MDE.

Third paragraph discusses the current state and potential drawbacks of MDE.

Forth paragraph introduces the notion of CORE.
\todo[inline]{and how it uses aspect-oriented modelling techniques to allow complex models to be built by incrementally composing smaller, simpler models. Mention that for now only design models, i.e., models of one development phase, are supported.}

Fifth paragraph continues with TouchCORE and its current state.
\todo[inline]{maybe not needed}

Build on the motivation of having UCM as an additional model for TouchCORE.
\todo[inline]{i.e., UCMs are scenario models, typically built earlier than design models. Aspect-oriented composition has already been investigated for UCMs (reference Gunter's Ph.D. and papers). The goal of this thesis is to concernify UCMs, i.e., determine how UCMs can be integrated with the concepts of CORE, and how the weaving needs to be adapted to fit with CORE. The ultimate goal being to investigate whether real MDE, i.e., software development with models at multiple levels of abstraction and model transformations that connect them, is compatible with CORE.}

Last paragraph leads the reader to the remaining chapters.
