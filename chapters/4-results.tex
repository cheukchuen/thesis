\todo[inline]{Is there a lot of interesting details to say about your implementation in TouchCORE / EMF / Java? If yes, then keep this chapter. If not, you could add a section about the implementation to the validation chapter, as an implementation is a form of validation (in the sense that if you implement your weving algorithm, and can run it on models to compose them, and the result is "correct", i.e., the algorithm terminates and the resulting model is as expected, then your algorithm must at least not be completely flawed :)}

The definition of UCM metamodel and the specification of weaving algorithm described in previous chapter provide the foundation for the implementation of UCM in TouchCORE. In this chapter, we illustrate the realization of scenario models in TouchCORE through the use of UCM notation in section \ref{sec:4.1}. Then we describe the weaving process for extensions and reuses in section \ref{sec:4.2}.

\section{Scenario Model in TouchCORE} \label{sec:4.1}

Lots of figures here.

\section{Model Weaving with UCMs} \label{sec:4.2}

\todo[inline]{Intro}

\subsection{Model Extension}

\subsection{Model Reuse}
